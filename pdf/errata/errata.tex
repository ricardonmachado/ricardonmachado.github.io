\documentclass[12pt,a4paper]{report}
\usepackage[utf8]{inputenc}
\usepackage{indentfirst}
\usepackage{amssymb,amsmath,amsthm,amsfonts,amscd}	%s�mbolos e caracteres especiais
\usepackage{empheq}
\usepackage{latexsym}

\usepackage{graphicx,graphics,epsfig}								%para inserir figuras
\usepackage{graphpap}										%pictures
\usepackage{float}
\usepackage{xcolor}
\usepackage[brazil]{babel}

\usepackage[T1]{fontenc}
\usepackage{graphicx}
\usepackage[normalem]{ulem}
\usepackage[a4paper,top=20mm,bottom=20mm,left=20mm,right=20mm]{geometry}
\usepackage{textcomp}
\usepackage{amsthm}
\usepackage[normalem]{ulem}
\usepackage{multicol}


\newtheorem{q}{}


\usepackage{array}




\begin{document}
\pagenumbering{arabic}
\thispagestyle{empty}










\begin{center}
ERRATA 
\end{center}

\noindent
Silva, L., Santos, M., Machado, R. \textit{Elementos de Computação Matemática com SageMath}. SBM. Rio de Janeiro. 2019.


% folha, linha, onde se lê, leia-se

\vspace{1cm}

\centering
\setlength{\extrarowheight}{1.5pt}
\begin{tabular}{|c|c|c|c|}
\hline
 ~~ Folha ~~ & ~~ Linha ~~ & ~~~~~~~~~~~~ Onde se lê ~~~~~~~~~~~~ & ~~~~~~~~~~~~ Leia-se ~~~~~~~~~~~~ \\
\hline \hline
 303 & 1 & classe $C^1$  & classe $C^2$ \\ \hline
 & & & \\ \hline
\end{tabular}

% 
% {
% \setlength{\extrarowheight}{1.5pt}
% \begin{tabular}{|c|c|c|}
% \hline
% a & Row 1 & classe $C^2$ \\ \hline
% b & Row 2 & fdgtd\\ \hline
% 
% \end{tabular}
% }


% 
% \centering
% 
% \begin{tabular}{l l l }
% \toprule
% \textbf{Operação} & \textbf{Sintaxe} & \textbf{Exemplo}  \\
% \midrule
% $\displaystyle\lim\limits_{x\to a}f(x)$ &  \verb|limit(f, x=a)|\index[metodo]{{\tt limit}} &   \ref{example1}  \\
% 
% $\displaystyle\lim\limits_{x\to +\infty}f(x)$ &\verb|limit(f, x=+infinity)| &   \ref{example6} \\
% 
% $\displaystyle\lim_{x\to -\infty}f(x)$ & \verb|limit(f, x=-infinity)| &   \ref{example6}\\
% 
% $\displaystyle\lim_{x\to a^+}f(x)$ & \verb|limit(f, x=a, dir='+')| &   \ref{example7}\\
% $\displaystyle\lim_{x\to a^-}f(x)$ & \verb|limit(f, x=a, dir='-')|  &   \ref{example8}\\
% \bottomrule
% \end{tabular}
% \caption{Operações de limite de funções.}\label{limite_de_funções}
% \end{table}


\end{document}