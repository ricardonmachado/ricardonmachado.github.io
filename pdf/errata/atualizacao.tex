\documentclass[12pt,a4paper]{report}
\usepackage[utf8]{inputenc}
\usepackage{indentfirst}
\usepackage{amssymb,amsmath,amsthm,amsfonts,amscd}	%s�mbolos e caracteres especiais
\usepackage{empheq}
\usepackage{latexsym}

\usepackage{graphicx,graphics,epsfig}								%para inserir figuras
\usepackage{graphpap}										%pictures
\usepackage{float}
\usepackage{xcolor}
\usepackage[brazil]{babel}

\usepackage[T1]{fontenc}
\usepackage{graphicx}
\usepackage[normalem]{ulem}
\usepackage[a4paper,top=20mm,bottom=20mm,left=20mm,right=20mm]{geometry}
\usepackage{textcomp}
\usepackage{amsthm}
\usepackage[normalem]{ulem}
\usepackage{multicol}


\newtheorem{q}{}


\usepackage{array}




\begin{document}
\pagenumbering{arabic}
\thispagestyle{empty}










\begin{center}
ATUALIZAÇÃOES 
\end{center}

\noindent
Silva, L., Santos, M., Machado, R. \textit{Elementos de Computação Matemática com SageMath}. SBM. Rio de Janeiro. 2019.


% folha, linha, onde se lê, leia-se

\vspace{1cm}


Na versão 9.0 do SageMath foi feita uma grande atualização. O python 2 foi substituído pelo python 3. O python 2 ainda era utilizado em muito softwares, mas os desenvolvedores descidiram parar de atualizar esta versão para ajudar os desenvolvedores do python 3. Assim em 2020 o python 2 perdeu o suporte e todos os softwares que usam python precisaram se adaptar ao python 3.

A lógica da linguagem continua a mesma, mas em alguns casos, um pouco de modificação na sintaxe será necessária para que os códigos funcionem na versão 3 ou seja, no SageMath 9.0 em diante.

Essa modificação, apesar de proporcionar o desconforto da adaptação, visa um aprimoramento no desempenho e otimização de memória. 

Vamos listar todos os comandos incluídos no livro que sofreram alguma alteração, mesmo que não tenha deixado de funcionar.

\begin{itemize}
 \item [i.] O \verb|print| agora é um método. Então, onde antes se escrevia: 
 \begin{verbatim}
 sage: print 'Um elefante'
  'Um elefante'
 \end{verbatim}
Agora se escreve:
 \begin{verbatim}
 sage: print('Um elefante')
  'Um elefante'
 \end{verbatim}
 
 \item [ii.] \verb|range|
 
 O método \verb|range| continua funcionando da mesma forma nos códigos. O que mudou foi que o software agora não cria uma lista inteira, fazendo com que se gaste menos memória ram. Mas a visualização mudou.
 Então, antes tinhamos:
 \begin{verbatim}
sage: range(5)
 [0, 1, 2, 3, 4]
 \end{verbatim}
Agora:
 \begin{verbatim}
sage: range(5)
 range(0, 5)
 \end{verbatim}
 Então, se quisermos ver a lista do \verb|range(5)|, fazemos o segunite.
 \begin{verbatim}
sage: list(range(5))
 [0, 1, 2, 3, 4]
\end{verbatim}
Mas quando o \verb|range| aparece no código, por exemplo, em um \verb|for|, nada muda, só que agora está mais otimizado, gastando menos memória ram.

\item [iii.] O \verb|raw_input| foi substituído por \verb|input|
\end{itemize}


\vspace{1cm}

Na tabela abaixo listamos a primeira ocorrência em que os métodos, que sofreram alguma alteração, aparecem no texto, mostrando como era antes e como ficou a partir do SageMath 9.0.

\centering
\setlength{\extrarowheight}{1.5pt}
\begin{tabular}{|c|c|c|c|}
\hline
 ~~ Folha ~~ & ~~ Linha ~~ & ~~~~~~~~~~~~ Onde se escreve ~~~~~~~~~~~~ & ~~~~~~~~~~~~ escreva ~~~~~~~~~~~~ \\
\hline \hline
 67 & 1 & \verb|print text1|  & \verb|print(text1)| \\ \hline
 72 & 4 & \verb|range(10)| &  \verb|list(range(10))|$^\star$ \\ \hline
 108 &  16 & \verb|raw_input| & \verb|input| \\ \hline
\end{tabular}


$*$ não confundir a visualização com a utilização do método. 
% 
% {
% \setlength{\extrarowheight}{1.5pt}
% \begin{tabular}{|c|c|c|}
% \hline
% a & Row 1 & classe $C^2$ \\ \hline
% b & Row 2 & fdgtd\\ \hline
% 
% \end{tabular}
% }


% 
% \centering
% 
% \begin{tabular}{l l l }
% \toprule
% \textbf{Operação} & \textbf{Sintaxe} & \textbf{Exemplo}  \\
% \midrule
% $\displaystyle\lim\limits_{x\to a}f(x)$ &  \verb|limit(f, x=a)|\index[metodo]{{\tt limit}} &   \ref{example1}  \\
% 
% $\displaystyle\lim\limits_{x\to +\infty}f(x)$ &\verb|limit(f, x=+infinity)| &   \ref{example6} \\
% 
% $\displaystyle\lim_{x\to -\infty}f(x)$ & \verb|limit(f, x=-infinity)| &   \ref{example6}\\
% 
% $\displaystyle\lim_{x\to a^+}f(x)$ & \verb|limit(f, x=a, dir='+')| &   \ref{example7}\\
% $\displaystyle\lim_{x\to a^-}f(x)$ & \verb|limit(f, x=a, dir='-')|  &   \ref{example8}\\
% \bottomrule
% \end{tabular}
% \caption{Operações de limite de funções.}\label{limite_de_funções}
% \end{table}


\end{document}
