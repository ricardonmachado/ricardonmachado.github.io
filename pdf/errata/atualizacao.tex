\documentclass[10pt,a4paper]{report}
\usepackage[utf8]{inputenc}
\usepackage{indentfirst}
\usepackage{amssymb,amsmath,amsthm,amsfonts,amscd}	%s�mbolos e caracteres especiais
\usepackage{empheq}
\usepackage{latexsym}

\usepackage{graphicx,graphics,epsfig}								%para inserir figuras
\usepackage{graphpap}										%pictures
\usepackage{float}
\usepackage{xcolor}
\usepackage[brazil]{babel}

\usepackage[T1]{fontenc}
\usepackage{graphicx}
\usepackage[normalem]{ulem}
\usepackage[a4paper,top=20mm,bottom=20mm,left=20mm,right=20mm]{geometry}
\usepackage{textcomp}
\usepackage{amsthm}
\usepackage[normalem]{ulem}
\usepackage{multicol}


\newtheorem{q}{}


\usepackage{array}


\usepackage{tcolorbox}% cria caixas para inserir texto
\tcbset{boxrule=0.2mm,colback=white} % configuração da caixa tcolorbox
\usepackage{verbatim}
\usepackage{enumitem} % Customize lists
%\usepackage{index}
% \usepackage{splitidx}
\usepackage{listings}
\lstset{
    escapeinside={(*}{*)}
}
%\usepackage[scale=0.85]{beramono}
% \usepackage{imakeidx}
%\usepackage[scale=10pt]{FiraMono}
%\usepackage[scale=0.85]{DejaVuSansMono}

%\usepackage[scale=0.85]{sourcecodepro}

\usepackage[scaled=0.80]{DejaVuSansMono}
%\usepackage[scale=0.85]{inconsolata}
%\usepackage[scale=0.82]{GoMono}
\newcommand{\un}{\underline{ }}
% \newcommand{\idx}[1]{\index[metodo]{{\tt #1}}}


\definecolor{mygray}{rgb}{0.42,0.42,0.42}
\usepackage{upquote}

\lstset{
  language=Python,
  showstringspaces=false,
  numberstyle= \scriptsize,
  numbers=left,
  xleftmargin=1em,
  xrightmargin=0.4em,
  numbersep=5pt,
  formfeed=\newpage,
  tabsize=4,
  commentstyle=\color{mygray}\itshape,
  %basicstyle=\footnotesize\fontfamily{fvm}\selectfont,%\small\ttfamily,
  basicstyle=\ttfamily,
  frame=single,
  literate= {ç}{{\c c}}1  {ú}{{\'u}}1 {í}{{\'i}}1 {â}{{\^a}}1 {ó}{{\'o}}1 {õ}{{\~o}}1 {á}{{\'a}}1 {ã}{{\~a}}1 {é}{{\'e}}1 {É}{{\'E}}1 {ê}{{\^e}}1 {ô}{{\^o}}1 {à}{{\`a}}1,
  morekeywords={srange, lambda}
}



\begin{document}
\pagenumbering{arabic}
\thispagestyle{empty}










\begin{center}
ATUALIZAÇÃOES 
\end{center}

\noindent
Silva, L., Santos, M., Machado, R. \textit{Elementos de Computação Matemática com SageMath}. SBM. Rio de Janeiro. 2019.


\vspace{1cm}

O SageMath 9.0 está usando o python 3, com isso algumas melhorias de otimização vieram, mas precisamos nos adaptar as pequenas alterações na parte da programação.

Em relação aos tópicos abordados no livro, três objetos sofreram alterações, dois na sintaxe e um no output. O que sofreu alteração no output, não tem problema no uso ao longo dos códigos.

Mudança na sintaxe:


Os dois objetos que sofreram alteração foram:

\verb|print| e \verb|raw_input|

Antes, para usar o print, bastava colocar o objeto a ser "printado" ao lado direito do print, por exemplo:
\begin{lstlisting}[frame=e, numbers=none, xleftmargin=0em]
sage: print 'a'
 'a'
\end{lstlisting}

Agora, precisamos colocar o objeto dentro do parênteses, por exemplo:
\begin{lstlisting}[frame=e, numbers=none, xleftmargin=0em]
sage: print('a')
 'a' 
\end{lstlisting}

a sintaxe do \verb|raw_input| mudou para \verb|input|.

Mudança no output:


O output do método \verb|range|, mudou.

Antes, o output do método range, era assim:
\begin{lstlisting}[frame=e, numbers=none, xleftmargin=0em]
sage: range(5)
 [0, 1, 2, 3, 4]
\end{lstlisting}

Agora, está assim:
\begin{lstlisting}[frame=e, numbers=none, xleftmargin=0em]
sage: range(5)
 range(0, 5)
\end{lstlisting}

Para obter a mesma visualização de antes, escrevemos o seguinte:

\begin{lstlisting}[frame=e, numbers=none, xleftmargin=0em]
sage: list(range(5))
 [0, 1, 2, 3, 4] 
\end{lstlisting}



Explicar que essa diferença veio para otimizar o software, fazendo que ele gaste menos memória ram.


Nas páginas seguintes, citamos as mudanças necessárias nos códigos do livro, para funcionar no SageMath 9.0.

\newpage

% folha, linha, onde se lê, leia-se

% \vspace{1cm}
% 
% 
% Na versão 9.0 do SageMath foi feita uma grande atualização. O python 2 foi substituído pelo python 3. O python 2 ainda era utilizado em muito softwares, mas os desenvolvedores descidiram parar de atualizar esta versão para ajudar os desenvolvedores do python 3. Assim em 2020 o python 2 perdeu o suporte e todos os softwares que usam python precisaram se adaptar ao python 3.
% 
% A lógica da linguagem continua a mesma, mas em alguns casos, um pouco de modificação na sintaxe será necessária para que os códigos funcionem na versão 3 ou seja, no SageMath 9.0 em diante.
% 
% Essa modificação, apesar de proporcionar o desconforto da adaptação, visa um aprimoramento no desempenho e otimização de memória. 
% 
% Vamos listar todos os comandos incluídos no livro que sofreram alguma alteração, mesmo que não tenha deixado de funcionar.
% 
% \begin{itemize}
%  \item [i.] O \verb|print| agora é um método. Então, onde antes se escrevia: 
%  \begin{verbatim}
%  sage: print 'Um elefante'
%   'Um elefante'
%  \end{verbatim}
% Agora se escreve:
%  \begin{verbatim}
%  sage: print('Um elefante')
%   'Um elefante'
%  \end{verbatim}
%  
%  \item [ii.] \verb|range|
%  
%  O método \verb|range| continua funcionando da mesma forma nos códigos. O que mudou foi que o software agora não cria uma lista inteira, fazendo com que se gaste menos memória ram. Mas a visualização mudou.
%  Então, antes tinhamos:
%  \begin{verbatim}
% sage: range(5)
%  [0, 1, 2, 3, 4]
%  \end{verbatim}
% Agora:
%  \begin{verbatim}
% sage: range(5)
%  range(0, 5)
%  \end{verbatim}
%  Então, se quisermos ver a lista do \verb|range(5)|, fazemos o segunite.
%  \begin{verbatim}
% sage: list(range(5))
%  [0, 1, 2, 3, 4]
% \end{verbatim}
% Mas quando o \verb|range| aparece no código, por exemplo, em um \verb|for|, nada muda, só que agora está mais otimizado, gastando menos memória ram.
% 
% \item [iii.] O \verb|raw_input| foi substituído por \verb|input|
% \end{itemize}
% 
% 
% \vspace{1cm}
% 
% Na tabela abaixo listamos a primeira ocorrência em que os métodos, que sofreram alguma alteração, aparecem no texto, mostrando como era antes e como ficou a partir do SageMath 9.0.
% 
% \centering{
% \setlength{\extrarowheight}{1.5pt}
% \begin{tabular}{|c|c|c|c|}
% \hline
%  ~~ Folha ~~ & ~~ Linha ~~ & ~~~~~~~~~~~~ Onde se escreve ~~~~~~~~~~~~ & ~~~~~~~~~~~~ escreva ~~~~~~~~~~~~ \\
% \hline \hline
%  67 & 1 & \verb|print text1|  & \verb|print(text1)| \\ \hline
%  72 & 4 & \verb|range(10)| &  \verb|list(range(10))|$^\star$ \\ \hline
%  108 &  16 & \verb|raw_input| & \verb|input| \\ \hline
% \end{tabular}
% 
% 
% $*$ não confundir a visualização com a utilização do método. 
% }
% 
% 
% \vspace{2cm}

\noindent
{ \large Capítulo 2 - SageMath: Primeiros Passos}

\begin{itemize}
 \item 
\end{itemize}



\vspace{1cm}
\noindent
{ \large Capítulo 3 - Introdução à Programação com Sage}
\begin{itemize}
 \item Adicionar \verb|list()|:
 \begin{itemize}
 \item Na página 71, adicione \verb|list()| no método \verb|range|. 
 
 
 \item Na página 78, linha -2, troque \verb|z| por \verb|list(z)|.
 
 \item Nas páginas 79, adicione \verb|list()| nos métodos \verb|zip| e \verb|map|.
 
 \item Na página 81, linhas 23 e 35, adicione \verb|list()| em \verb|dic_poligono.keys()| e \verb|dic_poligono.values()|.
 \end{itemize}


 
 \item Adicionar parênteses no método \verb|print|:
 \begin{itemize}
  
\item Nas páginas 83, 84, 85, 86, 88, 89, 90, 92 e 95, colocar parênteses no método \verb|print|.
 \end{itemize}
 
 
 
 \item Mudança na sintaxe:
\begin{itemize}
 \item Na página 108, linha 4 do código, troque \verb|raw_input| por \verb|input|.
\end{itemize}

\end{itemize}

\vspace{1cm}

\noindent
{\large Capítulo 4 - Matemática Elementar}

\begin{itemize}
 \item Na página 144,  colocar parênteses no método \verb|print|.
\end{itemize}



\vspace{1cm}
\noindent
{\large Capítulo 5 - Vetores, Matrizes e Álgebra Linear}


\begin{itemize}
 \item Na página 164, linha 5, o método \verb|adjoint| será trocado por \verb|adjugate| (o método \verb|adjoint| vai continuar funcionando por um tempo). {\bf não colocar}

 \item Na página 191, a função \verb|complemento_ortogonal_livro| deixou de funcionar, pois o SageMath 9.0 não está entendendo que os elementos de \verb|base_do_compl|, na linha 13, são vetores de \verb|V|. Para contornar este problema, adicione, entre as linhas 13 e 14, o código abaixo: {\bf ERRATA}
\begin{lstlisting}[frame=e, numbers=none, xleftmargin=0em]
temp=[]
for w in base_do_compl:
    temp.append(V(w))
base_do_compl = temp
\end{lstlisting}
\item Na página 215, na última linha, antes de \verb|A.is_similar(B)|, escreva:{\bf ERRATA}
\begin{lstlisting}[frame=e, numbers=none, xleftmargin=0em]
sage: A = A.change_ring(QQbar)
sage: B = B.change_ring(QQbar)
\end{lstlisting}

\end{itemize}


\vspace{1cm}
\noindent
{\large Capítulo 6 - Plot 2D}


\begin{itemize}
 \item Na página 235, Exemplo 6.16, o SageMath 9.0 não está gostando do latex. {\bf não colocar}

 \item Na página 247, Exemplo 6.28, O limite inferior 0, precisa ser trocado por um valor positivo. Do jeito que está ainda funciona. {\bf não colocar}
\end{itemize}



\vspace{1cm}
\noindent
{\large Capítulo 7 - Plot 3D}


Nada mudou 


\vspace{1cm}
\noindent
{\large Capítulo 8 - Cálculo Diferencial e Integral}


\begin{itemize}
 \item Na página 313, Exemplo 8.34, não está terminando o cálculo.
 \item Na página 335, Colocar parênteses no print.
\end{itemize}


\vspace{1cm}
\noindent
{\large Capítulo 9 - Equações Diferenciais Ordinárias}


\begin{itemize}



\item Na página 383, linha -2, \verb|color=hue(1./3*i)| está deixando tudo branco.

\end{itemize}


\vspace{1cm}
\noindent
{\large Capítulo 10 - Soluções de Problemas com Sage}


\begin{itemize}
 \item Na página 400, colocar parênteses nos prints. 
 \item Na página 400, última linha, trocar \verb|ticks=[Ox, Oy]| por \verb|ticks=[list(Ox),list(Oy)]|.
 \item Na página 401, colocar parênteses no print da linha 11 do código. 
 \item Na página 402, colocar parênteses no print da linha 2 do código. 
\end{itemize}



% 
% {
% \setlength{\extrarowheight}{1.5pt}
% \begin{tabular}{|c|c|c|}
% \hline
% a & Row 1 & classe $C^2$ \\ \hline
% b & Row 2 & fdgtd\\ \hline
% 
% \end{tabular}
% }


% 
% \centering
% 
% \begin{tabular}{l l l }
% \toprule
% \textbf{Operação} & \textbf{Sintaxe} & \textbf{Exemplo}  \\
% \midrule
% $\displaystyle\lim\limits_{x\to a}f(x)$ &  \verb|limit(f, x=a)|\index[metodo]{{\tt limit}} &   \ref{example1}  \\
% 
% $\displaystyle\lim\limits_{x\to +\infty}f(x)$ &\verb|limit(f, x=+infinity)| &   \ref{example6} \\
% 
% $\displaystyle\lim_{x\to -\infty}f(x)$ & \verb|limit(f, x=-infinity)| &   \ref{example6}\\
% 
% $\displaystyle\lim_{x\to a^+}f(x)$ & \verb|limit(f, x=a, dir='+')| &   \ref{example7}\\
% $\displaystyle\lim_{x\to a^-}f(x)$ & \verb|limit(f, x=a, dir='-')|  &   \ref{example8}\\
% \bottomrule
% \end{tabular}
% \caption{Operações de limite de funções.}\label{limite_de_funções}
% \end{table}


\end{document}
